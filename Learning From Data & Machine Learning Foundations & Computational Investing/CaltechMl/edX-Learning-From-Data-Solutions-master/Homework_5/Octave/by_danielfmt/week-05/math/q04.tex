\documentclass{article}

\usepackage{amsmath}
\usepackage{numprint}

\author{Daniel Fernandes Martins (danielfmt)}
\title{Question \#4 Solution}

\begin{document}

\maketitle

\section{Partial Derivative of $E(u,v)$}

In this exercise, we are asked to compute the partial derivative
$\frac{\partial E}{\partial u}$ of an error function that defines a
hypothetical error surface.

\begin{equation*}
\frac{\partial}{\partial u} E(u,v) = (ue^v -2ve^{-u})^2
\end{equation*}

Let's define the partial derivative of each term, step-by-step. First, we apply
the \textit{power rule}, which says $\frac{\partial}{\partial x} x^n = n \cdot x^{n-1}$:

\begin{equation*}
\frac{\partial}{\partial u} E(u,v) = 2(ue^v -2ve^{-u})^1 \\ 
\frac{\partial}{\partial u} (ue^v -2ve^{-u})
\end{equation*}

Notice that we also applied the \textit{chain rule} to the squared term since
it contain terms of $u$.

Now, the term $ue^v$, which is $u$ multiplied by a constant, is really simple
to differentiate:

\begin{equation*}
\frac{\partial}{\partial u} (ue^v) = 1 \cdot e^v = e^v
\end{equation*}

Finally, to differentiate a term that looks like $e^u$ we just need to multiply
it by the derivative of $u$, which is 1 or -1 depending on the sign:

\begin{equation*}
\frac{\partial}{\partial u} (-2ve^{-u}) = -2ve^{-u} \cdot -1 = 2ve^{-u}
\end{equation*}

Putting everything together, we have the solution:

\begin{equation*}
\frac{\partial}{\partial u} E(u, v) = \\
2(ue^v -2ve^{-u})(e^v + 2ve^{-u})
\end{equation*}

\end{document}
