\documentclass{article}

\usepackage{amsmath}
\usepackage{numprint}

\author{Daniel Fernandes Martins (danielfmt)}
\title{Question \#10 Solution}

\begin{document}

\maketitle

\textbf{Disclaimer.} This is the reasoning I used to solve the problem; it
may be wrong though. This is intended just as food for thought.

\section{Parameters Of Neural Network}

Given a neural network with 10 input units (including the constant $x_0^{(0)}$
unit), one output unit, and 36 hidden units (including the necessary number of
constant units for a fully-connected network), this question asks what is the
maximum possible number of weights that such a network can have.

\subsection{Solution By Trial And Error}

I found the solution by trying different configurations until I got a number of
weights equal to the highest number given in the alternatives.

The configuration I found has $L=3$; $d^{(0)}=10$, $d^{(1)}=22$, $d^{(2)}=14$,
$d^{(3)}=1$. The number of weights in this network is given by:

\begin{equation*}
(21 \cdot 10) + (13 \cdot 22) + (14) = 510
\end{equation*}

\end{document}
